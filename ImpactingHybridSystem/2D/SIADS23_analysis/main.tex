\documentclass{article}

% Language setting
% Replace `english' with e.g. `spanish' to change the document language
\usepackage[english]{babel}
\usepackage{comment}
% Set page size and margins
% Replace `letterpaper' with`a4paper' for UK/EU standard size
\usepackage[letterpaper,top=2cm,bottom=2cm,left=3cm,right=3cm,marginparwidth=1.75cm]{geometry}

% Useful packages
\usepackage{amsmath}
\usepackage{amssymb}
\usepackage{mathtools}
\newtheorem{theorem}{Theorem}[section]
\newtheorem{lemma}[theorem]{Lemma}
\newtheorem{proof}{Proof}[theorem]

\usepackage{paracol}
\usepackage{graphicx}
\usepackage{subfig}
%\usepackage{subcaption}
\usepackage[colorlinks=true, allcolors=blue]{hyperref}
\usepackage{cleveref}
%
\usepackage{caption}
\title{Revision of the DIB in 2D }
\author{Peter Tang}

\begin{document}
\maketitle

\begin{abstract}
Your abstract.
\end{abstract}


%> the first section to raise the 2D problem
\section{Re-analysis of the 2D Hopf like bifurcation in non-smooth systems}
\subsection{2D impacting hybrid system}
\label{sec:2D_IHS}

To start from the complexity analysis of general $n$ dimensional piecewise system and to compare the different aspects of the normal form.

From the \emph{SIADS23}, we reformulate the DIB  problem in 2D impacting hybrid systems.

\begin{lemma}[Topological equivalent lemma]
	conclusion is: the Li\'enard form with $C = [1,0]$ and $C^TB = 0$ can describe the general case of the 2D boundary equilibrium bifurcation of impacting hybrid systems.
	The form is given by
	$
	A = \begin{bmatrix}
		\lambda_1 + \lambda_2 & 1 \\
		-\lambda_1 \lambda_2 & 0
		\end{bmatrix}
	$
	and 
	$
	B = [0, (1+r)]^{\top},
	$
where $r>0$.

For focus case,
$
\lambda_1 = \alpha + \omega i, \lambda_2 = \alpha - \omega i.
$
and the reset map is given by 
$$
\mathcal{R} =Px^-.
$$
where 
$ P= (I - BCA)$.

\label{lemma:2D canonical form lemma}
\end{lemma}

\begin{lemma}
	For a general linearized impacting hybrid system, 
	\begin{equation}
		\begin{cases}
			F  &= A_0 x + N_0 u \\
			H(x) &= C_0^{\top}x  \\
			R(x^-) &= x^- - B_0 C_0^{\top} A_0 x^-
		\end{cases}
	\end{equation}
where 
\begin{equation}
	C_0^{\top} B_0 =0.
	\label{eq: RMP tangency condition}
\end{equation}
The observability matrix for this system is 
\begin{equation}
	\mathcal{O}_0 = [C_0, A_0^{\top} C_0, \cdots , (A_0^{\top})^{n-1}C_0],
\end{equation}
with full rank.
%
If we say the basis of the $\operatorname{ker}(C_0)$ is $\zeta_i, i = 2,\dots,n-1$, $B_0$ can be characterized  as 
\[
 B_0  = b_1  C_0 + \sum_{i=2}^{n} b_i \zeta_i
\]
and $b_1=0$ according to \ref{eq: RMP tangency condition}.
Under a linear transformation, the observability matrix maintains the same. So we can always use a transformation $x = P_0 y$
\[
 P_0 = [C_0; \operatorname{ker}(C_0)],
\]
the system can be converted to 
\begin{equation}
	\begin{cases}
		F  &= A_1 y + N v  \\
		H(y) &= \mathbf{e_1}^{\top} y \\
		 R(y^-) &= y^- - B_1 \mathbf{e_1} A y^- ,
	\end{cases}
\end{equation}
where $B_1 = P_0^{\top} B_0 = [0,b_2, \cdots, b_n]^{\top}$ and $A_1 = P_0^{\top} A_0 P_0$.

Now the observability matrix in the new coordinates should be 
$\mathcal{O} = [\mathbf{e_1}, A_1^{\top}\mathbf{e_1}, \cdots , (A_1^{\top})^{n-1}\mathbf{e_1}],$ which is also full rank.
Now the system is with $n^2$ parameters in the matrix $A_1$.

\label{lemma: observable system in stabdard basis}
\end{lemma}

\begin{theorem}[Observable canonical form]
	The system in \ref{lemma: observable system in stabdard basis} can be further transformed to observable canonical form.
	\label{th:OB_form_trans}
\end{theorem}

\begin{proof}
	Since $\mathcal{O}$ is full rank we know that  there is unique $z \in \mathcal{R}^n$ such that 
	$$
	z^{\top} [(A^{\top})^{n-1}\mathbf{e_1}, \cdots, A^{\top}\mathbf{e_1}, \mathbf{e_1}] = \mathbf{e_1}^{\top}
	$$ 
	while the $y = P\hat{y}$ with matrix 
	\[
	P = [A^{n-1} z , \cdots, Az, z]
	\]
	will transform the matrix to canonical form in \cref{lemma:2D canonical form lemma}.
\end{proof}
%> 
To derive the condition for the 2D BEB and the LCO's existence and stability. 

The existence condition for a LCO in 2D impacting hybrid system is given in [Hong Tang,et al. SIADS23] as
\begin{equation}
	 \operatorname{det}(\rm e^{A T} P - I ) =0
	 \label{eq:det_condition}
\end{equation}

\section{Focus case}
For a focus case, we have the general form 
\begin{equation}
	\begin{cases}
		F &= \begin{bmatrix}
			\alpha & \omega
			 \\
			-\omega & \alpha
		    \end{bmatrix} x + N_0 \mu
	  \\
	  H(x) &= C_0^{\top}x  
	  \\
	  R(x^-) &= x^- - B_0 C_0^{\top}A_0x^-
	\end{cases}
\end{equation}
where 
$C_0 = [\cos(\theta), \sin(\theta)]^{\top}$ 
and 
$B_0 = b_2 [-\sin(\theta), \cos(\theta)]^{\top}$.

Under the transformation of $x = P_0 y$ where 
$
P_0 = \begin{bmatrix}
	   \cos(\theta)   &  \sin(\theta) \\
	   - \sin(\theta) &  \cos(\theta)  
     \end{bmatrix}
$.
The system is redefined as 
\begin{equation}
	\begin{cases}
		F  & = A_1 x + N_1 \mu
		\\
		H(x) & = C_1^{\top}x  
		\\
		R(x^-) & = x^- - B_1 C_1^{\top} A_1x^-
	\end{cases}
\end{equation}
whereas 
$
A_1 = \begin{bmatrix}
	\alpha & \omega
	\\
	-\omega & \alpha
\end{bmatrix}
$
and 
$
C_1 = \mathbf{e}_1
$
and 
$
B_1 = [0,b_2]^{\top}
$.
Using the \cref{th:OB_form_trans} we can find another transformation $y = P \tilde{y}$ such that the system's canonical form is 
\begin{equation}
	\begin{cases}
		F  & = A \tilde{y} + N \mu
		\\
		H(\tilde{y})  & = C^{\top} \tilde{y} 
		\\
		R(\tilde{y}^-) & = \tilde{y}^- - B C^{\top}A\tilde{y}^-
	\end{cases}
\label{eq:OB_canonical_form_2D}
\end{equation}
where 
$
A = \begin{bmatrix}
	2 \alpha & 1
	\\
	-(\alpha^2 + \omega^2) & 0
\end{bmatrix}
$
and 
$
C = \mathbf{e}_1
$
and 
$
B = [0, \omega b_2]^{\top}
$, 
with 
$
z = [0, 1/\omega]^{\top},
$
and
$
P = \begin{bmatrix}
	1 & 0 
	\\
	\alpha/\omega & 1/\omega
\end{bmatrix}
$. For simplicity we define 
\[
\omega b_2 = (1 + r) 
\] 
and the system is now totally determined by $r, \alpha, \omega$.




Now, let us expand the  condition \cref{eq:det_condition} in \cref{eq:OB_canonical_form_2D} by 
\[
f(\mu, r,\hat{T}) = -r {\rm e}^{2 \mu \hat{T} } + 1 + \left[ (r - 1) \cos(\hat{T}) + \mu \sin(\hat{T}) (1 + r)\right] {\rm e}^{\mu \hat{T}}
\]
where we define $\mu = \alpha/\omega$ and the time is scaled by $\omega$.

If the solution space of $f(\hat{T}) =0 $ is not empty and we can find a set of LCO's solution given by 
$\hat{T} = T(\mu, r)$.
By solving $r$ form $f(\hat{T}) =0$  we have 
\begin{equation}
	r =\, \frac{\varphi(\hat{T}, \mu)}{\varphi(\hat{T}, -\mu)}  {\rm e}^{-2 \hat{T} \mu} 
	\label{eq:r_value_cond}
\end{equation}
and the $\varphi(\mu,T)$ is a function 
We introduce the auxiliary function
\begin{equation}
	\varphi(\tau,\mu)=1- e^{\mu \tau}(\cos{\tau}-\mu\sin{\tau}), 
	\; \partial \varphi/\partial{\tau}=e^{\mu\tau}(1+\mu^2)\sin{\tau}
\end{equation}
which can be also found in \cite{ANDRONOV1966443}.
%
The function $g$ is defined as 
\[
g(\tau,\mu) \coloneq \frac{\varphi(\tau, \mu)}{\varphi(\tau, -\mu)}  {\rm e}^{-2 \mu \tau}
\]

For a given $\mu$,  $r$ being in the range verifies the equivalent  existence condition 
\cref{eq:r_value_cond}. 

Moreover, the monotonicity of function $g(\tau)$ can be checked via confirming the derivative is semi-positive definitive.
% 
\begin{equation}
	\frac{\partial  g}{\partial  \tau} = \frac{g_1(\tau)}{\varphi(\tau,-\mu)^2} {\rm e}^{-2\mu \tau}
\end{equation}
in which 
\[
g_1(\tau) = \left[
\sin(\tau) \mu^2 + 2\mu \cos(\tau) - \sin(\tau)\right]
{\rm e}^{ - \mu \tau} 
+ 
\left[
-\sin(\tau) \mu^2 + 2\mu\cos(\tau) + \sin(\tau)
\right] {\rm e}^{\mu \tau} - 4\mu
\]
Obviously, the sign of $g_1(\tau)$ decides the sign of the derivative. We differentiate it again regarding $\tau$, we have
\[
g_2(\tau) \coloneq
\frac{\partial g_1}{\partial \tau} = -((\mu \sin(\tau) + \cos(\tau)) {\rm e}^{- \mu \tau} + (\mu \sin(\tau) - \cos(\tau)) {\rm e}^{ \mu \tau}) (\mu^2 + 1)
\]
and 
\[
g_3(\tau) \coloneq
\frac{\partial g_2}{\partial \tau} =(\mu^2 + 1)({\rm e}^{- \mu \tau} - {\rm e}^{ \mu \tau}) \sin(\tau).
\]

%> now talk about the initial condition
The starting point of the orbit is 
$
\hat{y} = [1,y_2]^{\top}
$
and the velocity is defined by
\begin{equation}
\mathcal{V}(\hat{y}) = \mathbf{e}_1^{\top} A \hat{y}
\end{equation}
A closed orbit should satisfy the returning condition in SIADS23 as 
\[
\mathbf{e}_1^{\top} {\rm e}^{A \hat{T}} \hat{y} = 1
\]
which gives 
\[
 \mathcal{V} = \omega {\rm e}^{-\mu \hat{T}} \frac{\varphi(\hat{T},\mu) }{\sin(\hat{T})}
\]
and 
\begin{equation}
	\frac{\partial \mathcal{V}}{\partial \tau} = \omega \frac{\varphi(\tau,-\mu)}{\sin^2(\tau)}
\end{equation}
The incoming velocity is given by
\[
v_{in} = -\omega {\rm e}^{\mu \hat{T}} \frac{\varphi(\hat{T},-\mu)}{\sin(\hat{T})}
\]
and 
\begin{equation}
	\frac{\partial \mathcal{V}}{\partial \tau} = -\omega \frac{\varphi(\tau,\mu)}{\sin^2(\tau)}
\end{equation}
% All the functions $g, g_1, g_2, g_3$ are defined 
 For $\mu < 0$, $g_3$
 

\appendix
\renewcommand{\theequation}{\Alph{section}.\arabic{equation}}
\renewcommand{\thesubsection}{\Alph{section}.\arabic{subsection}}
\renewcommand{\thesubsubsection}{\Alph{section}.\arabic{subsection}.\arabic{subsubsection}}
\renewcommand{\thefigure}{\Alph{section}.\arabic{figure}}
\renewcommand{\thetable}{\Alph{section}.\arabic{table}}
\clearpage

\clearpage
\bibliographystyle{alpha}
\bibliography{sample}

\end{document}